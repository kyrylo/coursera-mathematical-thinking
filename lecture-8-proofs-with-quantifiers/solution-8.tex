\documentclass{article}
\title{Solution 8}
\author{@kyrylo}

\usepackage{enumerate}
\usepackage{amsmath}
\usepackage{centernot}
\usepackage{amsfonts}

\begin{document}

\section*{Solutions to assignment 8}

\section{}

Theorem: ``There are integers $m, n$ such that $m^2 + mn + n^2$ is a perfect
square''.
\\
Proof: Let $A(m, n)$ denote ``$m^2 + mn + n^2$ is a perfect square''.
Hence, we're proving $(\exists m \in \mathbb{Z})(\exists n \in \mathbb{Z})[A(m, n)]$
Let's pick arbitrarily $m = 1, n = 0$. We are proving $A(1, 0)$, which is
$1^2 + 1 \cdot 0 + 0^2 = 1 + 0 + 0 = 1$. The acquired number $1$ is a perfect square and this
proves $A(m, n)$.

\section{}

Theorem: ``For any positive integer $m$ there is a positive integer $n$ such
that $mn + 1$ is a perfect square''
\\
Proof: Let $A(m, n)$ denote ``$mn + 1$ is a perfect square". Hence, we're
proving $(\forall m \in \mathbb{Z})(\exists n \in \mathbb{Z})[A(m, n)]$. Let $m$
be an arbitrary positive integer. Set $n = m + 2$. Then $mn + 1 = m \cdot (m +
2) + 1$ can be expressed as
\\
$m \cdot (m + 2) + 1 = m^2 + 2m + 1 = m^2 + m + m + 1 = (m + 1)(m + 1) = (m + 1)^2$
\\
This proves that $(m + 1)^2$ is a perfect square.

\section{}

Theorem: ``There is a quadratic $f(n) = n^2 + bn + c$ with positive integer
coefficients $b, c$, such that $f(n)$ is composite for all positive integers
$n$.''
\\
Proof: we are proving
\\
$(\exists b \in \mathbb{Z})(\exists c \in \mathbb{Z})(\forall n \in
\mathbb{Z})[(c > 0) \wedge (b > 0) \wedge (n > 0) \wedge \neg Prime(n^2 + bn +
c)]$ Let $b = 1, c = 2$, then $f(n) = n^2 + 2n + 1 = (n + 1)^2$. Since $(n +
1)^2$ always gives a perfect square, it means that the result is composite,
because it can be divided by $n + 1$ (and prime numbers can be divided only by 1 and
by itself). For example: $n = 36, f(n) = (n + 1)^2 = (36 + 1)^2 = 37^2 =
1369$. So, $1369 / (n + 1)= 1369 / (36 + 1) = 37$.

\end{document}
