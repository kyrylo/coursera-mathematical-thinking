\documentclass{article}
\title{Solution 4}
\author{@kyrylo}

\usepackage{enumerate}
\usepackage{amsmath}
\usepackage{centernot}

\begin{document}

\section*{Solutions to assignment 4}

\section{}

\begin{tabular}{ | c | c | c | }
  \hline
  $\phi$ & $\psi$ & $\phi \Leftrightarrow \psi$ \\
  \hline
  T & T & T \\
  T & F & F \\
  F & T & F \\
  F & F & T \\
  \hline
\end{tabular}

\section{}

\begin{tabular}{ | c | c | c | c | c | c | }
  \hline
  $\phi$ & $\psi$ & $\neg \phi$ & $\phi \implies \psi$ & $\neg \phi \vee \psi$ &  $(\phi \implies \psi) \Leftrightarrow (\neg \phi \vee \psi)$ \\
  \hline
  T & T & F & T & T & T \\
  T & F & F & F & F & T \\
  F & T & T & T & T & T \\
  F & F & T & T & T & T \\
  \hline
\end{tabular}

\section{}

\begin{tabular}{ | c | c | c | c | c | c | }
  \hline
  $\phi$ & $\psi$ & $\neg \psi$ & $\phi \centernot\implies \psi$ & $\phi \wedge \neg\psi$ &  $(\phi \centernot\implies \psi) \Leftrightarrow (\phi \wedge \neg\psi)$ \\
  \hline
  T & T & F & F & F & T \\
  T & F & T & T & T & T \\
  F & T & F & F & F & T \\
  F & F & T & F & F & T \\
  \hline
\end{tabular}


\section{}

\begin{enumerate}[(a)]
\item
  \begin{tabular}{ | c | c | c | c | c | }
    \hline
    $\phi$ & $\psi$ & $\phi \implies \psi$ & $\phi \wedge (\phi \implies \psi)$ & $[\phi \wedge (\phi \implies \psi)] \implies \psi$ \\
    \hline
    T & T & T & T & T \\
    T & F & F & F & T \\
    F & T & T & F & T \\
    F & F & T & F & T \\
    \hline
  \end{tabular}
\item
  The statement gives truth in every case. Any combination of $\phi$ and $\psi$ is true.
\end{enumerate}

\section{}

\begin{enumerate}
\item $\phi \vee \psi$ means at least one from $\phi$ or $\psi$ must be true.
\item Thus $\neg(\phi \vee \psi)$ means none of $\phi$ or $\psi$ is true.
\item The conjoin $(\neg\phi) \wedge (\neg\psi)$ indicates that both $\phi$ and
  $\psi$ must be false.
\item Thus $(\neg\phi) \wedge (\neg\psi)$ is equivalent to $\neg(\phi \vee \psi)$.
\end{enumerate}

\section{}

\begin{enumerate}[(a)]
\item 34,159 is not a prime number.
\item Roses are not red and violets are not blue.
\item If there are no hamburgers, I'll not have a hot dog.
\item Fred will not go or he will play.
\item The number $x$ is either positive or not greater than 10.
\item We will not win the first game or the second.
\end{enumerate}

\section{}

\begin{tabular}{ | c | c | c | c | c | c | }
  \hline
  $\phi$ & $\psi$ & $\phi \Leftrightarrow \psi $ &  $\neg\phi$ & $\neg\psi$ & $(\neg\phi) \Leftrightarrow (\neg\psi)$ \\
  \hline
  T & T & T & F & F & T \\
  T & F & F & F & T & F \\
  F & T & F & T & F & F \\
  F & F & T & T & T & T \\
  \hline
\end{tabular}

\section{}

\begin{enumerate}[(a)]
\item
  \begin{tabular}{ | c | c | c | }
    \hline
    $\phi$ & $\psi$ & $\phi \Leftrightarrow \psi $ \\
    \hline
    T & T & T \\
    T & F & F \\
    F & T & F \\
    F & F & T \\
    \hline
  \end{tabular}

\item
  \begin{tabular}{ | c | c | c | c | c | }
    \hline
    $\phi$ & $\psi$ & $\theta$ & $\psi \vee \theta$ & $\phi \implies (\psi \vee \theta)$ \\
    \hline
    T & T & T & T & T \\
    T & T & F & T & T \\
    T & F & T & T & T \\
    T & F & F & F & F \\
    F & T & T & T & T \\
    F & T & F & T & T \\
    F & F & T & T & T \\
    F & F & F & F & T \\
    \hline
  \end{tabular}
\end{enumerate}

\section{}

\begin{tabular}{ | c | c | c | c | c | }
  \hline
  $\phi$ & $\psi$ & $\theta$ & $\psi \wedge \theta$ & $\phi \implies (\psi \wedge \theta)$ \\
  \hline
  T & T & T & T & T \\
  T & T & F & F & F \\
  T & F & T & F & F \\
  T & F & F & F & F \\
  F & T & T & F & T \\
  F & T & F & F & T \\
  F & F & T & F & T \\
  F & F & F & F & T \\
  \hline
\end{tabular}
\quad
\begin{tabular}{ | c | c | c | c | c | c | }
  \hline
  $\phi$ & $\psi$ & $\theta$ & $\phi \implies \psi$ & $\phi \implies \theta$ & $(\phi \implies \psi) \wedge (\phi \implies \theta)$ \\
  \hline
  T & T & T & T & T & T \\
  T & T & F & T & F & F \\
  T & F & T & F & T & F \\
  T & F & F & F & F & F \\
  F & T & T & T & T & T \\
  F & T & F & T & T & T \\
  F & F & T & T & T & T \\
  F & F & F & T & T & T \\
  \hline
\end{tabular}

\section{}

Suppose $\psi, \theta$ are true. Conjuction of them is also true, hence it is
obvious that $\phi \implies (\psi \wedge \theta)$ is true. Implication of $\psi,
\theta$ from $\phi$ is obvious, too. I suck at explanations.

\end{document}
